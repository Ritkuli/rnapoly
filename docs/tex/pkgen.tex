\section{Pkgen}
Pkgen is a tool used to generate and select pseudoknots. When used as a commandline program, it generates all possible combinations of two strands of length N and calculates the nearest-neighbor-energy of them using NUPACK. It takes N as an input and outputs a sorted list of the strands and their energies. Pkgen also implements an API of Python functions that can be called directly from another Python script to select pseudoknot strands from a list based on restrictions such as G-C content, prevented sequences or the energy between mismatching strands.
\subsection{Usage}
\begin{verbatim}
  usage: pkgen.py [-h] [-l LENGTH] [-c COUNT] output
\end{verbatim}
\begin{itemize}
  \item -h; show help message and exit
  \item output; The output path
  \item -l LENGTH; Pseudoknot length
  \item -c COUNT; Number of pseudoknots to generate
\end{itemize}

\subsection{Example}
Generating all length 5-pseudoknots:
\begin{verbatim}
./pkgen.py -l 5 5pk.txt
\end{verbatim}
Generating 5000 random length 7 pseudoknots:
\begin{verbatim}
./pkgen.py -l 7 -c 5000
\end{verbatim}
